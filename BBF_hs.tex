\documentclass[
10pt, % Main document font size
a4paper, % Paper type, use 'letterpaper' for US Letter paper
oneside, % One page layout (no page indentation)
%twoside, % Two page layout (page indentation for binding and different headers)
headinclude,footinclude, % Extra spacing for the header and footer
BCOR5mm, % Binding correction
]{scrartcl}

\usepackage{listings}
\usepackage{color}
%\usepackage{biblatex}

\definecolor{dkgreen}{rgb}{0,0.6,0}
\definecolor{gray}{rgb}{0.5,0.5,0.5}
\definecolor{mauve}{rgb}{0.58,0,0.82}

\lstset{frame=tb,
	language={},
	aboveskip=3mm,
	belowskip=3mm,
	showstringspaces=false,
	columns=flexible,
	basicstyle={\small\ttfamily},
	numbers=none,
	numberstyle=\tiny\color{gray},
	keywordstyle=\color{blue},
	commentstyle=\color{dkgreen},
	stringstyle=\color{mauve},
	breaklines=true,
	breakatwhitespace=true,
	tabsize=3
}

\usepackage{german}


%usepackage[utf8]{inputenc}
%\usepackage{geometry}
\usepackage[german,onelanguage,linesnumbered, ruled]{algorithm2e}
\SetAlFnt{\small}
\SetAlCapFnt{\large}
\SetAlCapNameFnt{\large}
%\usepackage{algpseudocode}


\input{structure.tex} % Include the structure.tex file which specified the document structure and layout

\hyphenation{Fortran hy-phen-ation} % Specify custom hyphenation points in words with dashes where you would like hyphenation to occur, or alternatively, don't put any dashes in a word to stop hyphenation altogether

%----------------------------------------------------------------------------------------
%	TITLE AND AUTHOR(S)
%----------------------------------------------------------------------------------------

\title{\normalfont\spacedallcaps{Projektaufgabe AE}} % The article title

\subtitle{Remove Duplicates - Spotify playlist cleaner} % Uncomment to display a subtitle

\author{\spacedlowsmallcaps{Raphael Drechsler}} % The article author(s) - author affiliations need to be specified in the AUTHOR AFFILIATIONS block

\date{} % An optional date to appear under the author(s)

%----------------------------------------------------------------------------------------

\begin{document}

%----------------------------------------------------------------------------------------
%	HEADERS
%----------------------------------------------------------------------------------------

\renewcommand{\sectionmark}[1]{\markright{\spacedlowsmallcaps{#1}}} % The header for all pages (oneside) or for even pages (twoside)
%\renewcommand{\subsectionmark}[1]{\markright{\thesubsection~#1}} % Uncomment when using the twoside option - this modifies the header on odd pages
\lehead{\mbox{\llap{\small\thepage\kern1em\color{halfgray} \vline}\color{halfgray}\hspace{0.5em}\rightmark\hfil}} % The header style

\pagestyle{scrheadings} % Enable the headers specified in this block

%----------------------------------------------------------------------------------------
%	TABLE OF CONTENTS & LISTS OF FIGURES AND TABLES
%----------------------------------------------------------------------------------------

%\maketitle % Print the title/author/date block
{ \centering
{ \par}\
 \linebreak
\linebreak 
\linebreak
\linebreak
\linebreak
%\centering
\includegraphics[width=0.55\columnwidth]{htwLogo} 
\linebreak
\linebreak
\linebreak
\linebreak 
 % inline
{\fontsize{14}{16}\selectfont \center Fakultät Informatik, Mathematik und\\Naturwissenschaften\\Studiengang Informatik Master\par}\
 \linebreak
{\fontsize{18}{20}\selectfont \center \textbf{Projektarbeit zur Vorlesung Computermusik}\par}\
{\fontsize{20}{22}\selectfont \center \textbf{BrandtBrauerFrick.hs} \par}\
\linebreak
\linebreak
\linebreak
\linebreak 
\linebreak
\linebreak 
\linebreak 
{\fontsize{14}{16}\selectfont  \begin{tabular}{rl}
 	\textbf{Autoren:} & Nico Mehlhose, Raphael Drechsler\\ 
 	\textbf{Abgabedatum:} & 01.02.2019 \\ 
 \end{tabular}
\par}
\par}
\pagebreak
\setcounter{tocdepth}{2} % Set the depth of the table of contents to show sections and subsections only

%\tableofcontents % Print the table of contents
%\listoffigures % Print the list of figures
%\listoftables % Print the list of tables




%----------------------------------------------------------------------------------------

\newpage % Start the article content on the second page, remove this if you have a longer abstract that goes onto the second page

%----------------------------------------------------------------------------------------
%	INTRODUCTION
%----------------------------------------------------------------------------------------
\section{Abstract}\
\textit{Abschnitt bearbeitet von: Raphael Drechsler}\\

\noindent \textbf{BrandtBrauerFrick.hs}

\noindent Brandt Brauer Frick ist ein Techno-Projekt aus Berlin.
Die Basis des Projekts bilden Klänge aus dem Instrumentarium der
klassischen Musik, welche anfangs gesampelt, später in einem zehnköpfigen
Ensemble auch live vorgeführt wurden.\cite{Wiki}\\ 

\noindent Ziel des Projektes:\\
Die Umsetzung des Songs ''Pretend'' von Brandt Brauer Frick entweder in
Tidal oder Euterpea. Eine online verfügbare Live-Aufführung \cite{YT1} soll dabei als Referenz dienen. Bei der Umsetzung soll auch Wert auf die Nachbildung der echten Instrumente und deren teilweise Zweckentfremdung gelegt werden.\\

\noindent Herausforderungen:
\begin{itemize}
	\itemsep0em
	\item Evaluation ob Tidal\cite{Tidal} oder Euterpea\cite{Euterpea} genutzt werden soll:
	\item Untersuchung der Frage ob klassische Klänge am ehesten in Euterpea oder
	Tidal nutzbar sind. (Durch repetitiven Charakter des Liedes würde sich Tidal zur
	Live-Vorführung eignen)
	\item Analyse der einzelnen musikalischen Bausteine und deren Implementierung.
	\item Zusammenfügen der erarbeiteten Bausteine zu einer Performance.
\end{itemize}

\section{Umsetzung in Tidal oder Euterpea}\
\textit{Abschnitt bearbeitet von: Nico Mehlhose}\\

\noindent Dieses Thema soll sich um die Evaluation zwischen Tidal und Eutherpea handeln.\\
{\color{red}\textbf{TODO}}Was ist Tidal\cite{Tidal} (dazu SuperCollider\cite{SC} erklären) , was ist Euterpea\cite{Euterpea}?\\
Unsere Entscheidung Tidal zu nehmen beruht gewiss nicht auf einer zufälligen Entscheidung. In diese Entscheidung ist der Programmieraufwand, vorhandenen Informationen
und die Möglichkeit den Synthesizer zu erweitern.\\
Bei dem Programmieraufwand wird sehr schnell klar, dass durch das Lied \textit{Pretent} von BrandBrauerFrick Tidal besser geeignet ist als Euterpea. Der erste Gesichtspunkt
der betrachtet wurde ist die Repetetivität des Songs, welcher in Euterpea zwar auch umsetzbar ist aber in Tidal von Anfang an gegeben ist, da Tidal die Sounds immer in einem
Loop abspielt. Bei den vorhandenen Informationen stellt sich heraus, dass es keine Offiziellen Notenblätter für das Lied Onlinegibt, wodurch Euterpea etwas an Bedeutung verliert, da Euterpea für genaue Notenbestimmungen perfekt geeignet wäre. Da dieser Fakt aber nicht vorliegt, kann das selbe Maß an Genauigkeit auch mit Tidal erreicht werden.\\
Der letzte und für uns wichtigste Punkt war die Erweiterbarkeit der Sounds. Die Wichtigkeit darin besteht in der entfremdeten Benutzung der Musikinstrumente in dem Lied.
In Eutherpea haben wir nach einiger Recherche keinen weg gefunden Sounds hinzuzufügen um diese später zu verwenden. In Tidal allerdings existiert diese Möglichkeit mittels
des Befehl \textit{}. Mit diesem Befehl lässt sich ein Verzeichnis in Tidal integrieren.
%~dirt.loadSoundFiles("full/path/to/directory/*") noch in textit einfügen
\section{Eigenschaften des Stücks Pretend und dessen globale Struktur}\
\textit{Abschnitt bearbeitet von: Raphael Drechsler}\\

\noindent Die in der Live vorgeführte Version \cite{YT1} hat eine ungefähre Dauer von 7 Minuten, 15 Sekunden. Die Angabe erfolgt ungefähr, da die Aufnahme nicht mit dem ersten Takt beginnt\\
Per Gehör ließ sich feststellen, dass das Stück in der Tonart Gm steht.\\
Über ein BPM-Measuring-Tool \cite{tempo} wurde ein Tempo von 130bpm ermittelt. In Tidal wird somit der folgende Code zur Tempo-Einstellung benötigt. 
\begin{lstlisting}
setcps (130/60/4)
\end{lstlisting}

\noindent Die Globale Struktur des Liedes, also die Zeitliche Abfolge der Figuren der einzelnen Instrumente, wurde per Gehör analysiert. Dabei wurde ebenfalls die Live-Version des Liedes als Untersuchungsgegenstand verwendet. Um das Ergebnis zu visualisieren, wurden in Logic Pro\cite{Logic} (einer digital Audio-Workstation der Firma Apple) für die jeweiligen Figuren leere MIDI-Regionen innerhalb der 237 Takte erzeugt.
Anschließend wurde das Resultat per Screenshot aufgenommen und die einzelnen Figuren mit F1,F2,... für die jeweilige Figur beschriftet.\\
Für die ersten vier Takte wurde bei der Erstellung der MIDI-Regionen eine Annahme getroffen.

\begin{figure}[h]
	\centering 
	\includegraphics[width=0.99\columnwidth]{GlobaleStrukturMarkiert} 
	\caption{Globale Struktur dargestellt als leere MIDI-Regionen in Logic Pro mit Beschriftung der Figuren}
\end{figure}

\noindent Es wurde ebenfalls versucht die Abfolge mithilfe eines freien Notations-Programmes in einer Partitur darzustellen. Jedoch erwies sich die obige Darstellung als kompakter und ausreichend.

\section{Analyse und Synthese der einzelnen Instrumente}\
\textit{Abschnitt bearbeitet von: Raphael Drechsler}\\

\noindent Im folgenden Abschnitt sollen die zehn Instrumente synthetisiert werden. Wie in der globalen Struktur (siehe Abb. 1) zu erkennen ist, existieren in den meisten Fällen pro Instrument mehrere Figuren. Die folgenden Arbeitsschritte sollen daher pro Instrument und Figur erfolgen.

\paragraph{Analyse der gespielten Tonhöhen und Rhythmen.}  Diese Analyse erfolgt per Gehör. Untersucht wird dabei die Live-Version\cite{YT1} des Stückes. Für Parts, die besonders schwierig herauszuhören sind, da z.B. das Instrument nur sehr schwer hörbar ist, werden zusätzlich eine ähnliche Studio-Version\cite{YT2} sowie eine früher Variante\cite{YT3} des Liedes für die Untersuchung herangezogen.	Das Ergebnis der Analyse soll hier als Text, welcher die Figur beschreibt und/oder mithilfe von Noten dargestellt werden. 

\paragraph{Synthese von Tonhöhe und Rhythmus} Die analysierten Tonhöhen und Rhythmen sollen als Tidal Code umgesetzt werden. Dabei wird kein gesteigerter Wert auf die Auswahl eines passenden Klanges gelegt. Im Vordergrund der Betrachtung steht, dass der umgesetzte Code den analysierten Tonhöhen und Rhythmen entspricht. Vereinzelt soll hierbei auf zu überwindende Herausforderungen und genutzte Funktionen eingegangen werden.

\paragraph{Klanganalyse} Per Gehör soll das Klangbild des Instrumentes in der speziellen Figur sowie die Wirkung, die durch die Figur beim Hörer erzeugt wird untersucht werden. 

\paragraph{Anpassung der Klangsynthese} Unter Berücksichtigung des Analysierten Klangbildes und des bereits vorliegenden Tidal-Codes soll nun die Art der Klangsynthese derart angepasst werden, dass die Beschriebene klangliche Wirkung erzielt wird. Mögliche Arten der Anpassung sind dabei die Auswahl von Tidal-Instrumenten, Einbinden von Samples sowie Erstellen eines SuperCollider-Instrumentes.\\

\noindent Auf diesem Wege sollen die Figuren aller Instrumente als ausführbarer Tidal-Code mit erwünschter klanglicher Wirkung entstehen.
Das Verbinden der einzelnen Figuren zu einem live vorführbaren Stück soll in \textit{Kapitel 6 - Performance} beschrieben werden. 


%
%\subsection{Instrument 0: Was ist pro Instrument TODO?}
%{\color{red}\textbf{TODO}}: Nach Bearbeitung Hilfskapitel entfernen.\\
%
%Raph.\\
%Welche Figuren?
%- Welche Wirkung?
%- Welche Noten?\\
%
%Nico\\
%Wie klingt das Instrument?\\
%- Wie klingt das live? Einzelne Bestandteile? (Marimba gespielt mit Holzsticks und verschiedene Kuhglocken)
%- Wie klingt das in welcher Figur? (zB. BD laut, leise)
%- Welchen Klang wählen (evaluation - SD-Instrument nutzbar?, WAV suchen/selber aufnehmen, Instrument coden)\\

\subsection{Instrument 1: Schlagzeug}
\subsubsection{Figuren}
\textit{Abschnitt bearbeitet von: Raphael Drechsler}\\

\noindent \textbf{Figur 1}\\
Herausgehört wurde das folgende Muster. Die Note \textit{A} steht dabei für die Base-Drum, Note \textit{Dis} für die High-Hat.
\begin{figure}[h]
	\centering 
	\includegraphics[width=0.3\columnwidth]{Drum_Fig1} 
	\caption{Schlagzeug Figur 1}
\end{figure}

\noindent Um in Tidal pro Takt eine Figur mit 4 Schlägen auf die Base-Drum und 4 Schlägen auf die High-Hat im Wechsel zu realisieren, lässt sich eine Kombination von Gruppierung und Wiederholung\cite{tid1} verenden:
\begin{lstlisting}
d1 $ sound "[bd hh]*4"
\end{lstlisting}

\noindent \textbf{Figur 2}\\
Herausgehört wurde eine Figur über 8 Takte. Dabei werden in Takt 3,7 und 8 wie nachfolgend notiert Fills auf der High-Hat gespielt.\\
\begin{figure}[h]
	\centering 
	\includegraphics[width=0.8\columnwidth]{Drums_Fig2A} 
	\includegraphics[width=0.8\columnwidth]{Drums_Fig2B} 
	\caption{Schlagzeug Figur 2}
\end{figure}

\noindent Die Umsetzung in Tidal der einzelnen Takte ohne Fills erfolgt analog zu Figur 1. In den Takten mit Fills werden auf einige Zählzeiten Base-Drum und High-Hat gleichzeitig gespielt. Dies kann in Tidal durch die Nutzung von   \verb1stack1 \cite{tid2} umgesetzt werden. Über den \verb1cat1-Ausdruck\cite{tid3} werden die acht Figuren hintereinander gespielt, was den Cycle auf die gewünschte Länge von acht Takten verlängert. Es ergibt sich folgender Code.

\begin{lstlisting}
d1 $  cat [
sound "[[bd hh]*4]",
sound "[[bd hh]*4]",
stack [ sound "[bd ~]*4", sound "[~ hh ~ hh][~ [hh hh] hh hh]" ],
sound "[[bd hh]*4]",
sound "[[bd hh]*4]",
sound "[[bd hh]*4]",
stack [ sound "[bd ~]*4", sound "[~ hh ~ hh][~ [hh hh] hh hh]" ],
stack [ sound "[bd ~]*4", sound "[~ hh ~ hh][~ [hh hh] [hh hh] hh]" ]
]
\end{lstlisting}

\noindent \textbf{Figur 3}\\
Herausgehört wurde die folgende Figur mit zwei Takten Länge. Die Note \textit{Fes} steht dabei für eine geöffnete High-Hat.
\begin{figure}[h]
	\centering 
	\includegraphics[width=0.5\columnwidth]{Drums_Fig3} 
	\caption{Schlagzeug Figur 1}
\end{figure}

\noindent In Tidal wurde die Figur hintereinander in Gruppierungen geschrieben und per \verb1slow 21\cite{tid4} auf die Länge von zwei Takten gestreckt.
\begin{lstlisting}
d1 $  slow 2 $ sound "[bd hh bd hh][bd [hh ho] bd hh] [bd hh bd hh][bd [hh hh] bd hh]"
\end{lstlisting}

\noindent \textbf{Figur 4}\\
Analog zu Figur 1. Dazu kommen zyklische Bewegung auf der Rim (Rand der Snare-Drum).
Die Figur ließ sich nur schwer durch Raushören bestimmen. Es wurden daher 5 Schläge auf die Rim pro Takt als Annahme getroffen, wobei aller 2 Takte der letzte Schlag ausgelassen wird.\\

\noindent Dies realisiert der folgende Tidal-Code.
\begin{lstlisting}
d1 $ stack [
sound "[bd hh]*4",
cat[sound "[rm rm rm rm rm]", sound "[rm rm rm rm ~]"] 
]
\end{lstlisting}

\noindent \textbf{Figur 5}\\
Nur die zyklische Rimclick-Bewegung aus Figur 4.
\begin{lstlisting}
d1 $ cat[sound "[rm rm rm rm rm]", sound "[rm rm rm rm ~]"]
\end{lstlisting}

\noindent \textbf{Figur 6 und 7}\\
Figur 6 wie Figur 4 und Figur 7 wie Figur 3. Jeweils kräftiger gespielt. Dies wird später in der Performance mit einem \verb1# gain1-Ausdruck\cite{tid5} zum Regulieren der Lautstärke des gespielten Samples umgesetzt.\\

\noindent \textbf{Figur 8}\\
Wie Figur 4 aber ohne Base-Drum.
\begin{lstlisting}
d1 $ stack [
sound "[~ hh]*4" ,
cat[sound "[rm rm rm rm rm]", sound "[rm rm rm rm ~]"] 
]
\end{lstlisting}

\subsubsection{Klangbild}
\textit{Abschnitt bearbeitet von: Nico Mehlhose}\\

\noindent Bestandteile des Schlagzeuges: herkömmliches Schlagzeug\\
Art der Synthetisierung: Da Bass Drum und High Head normal gespielt werden können die Sounds aus dem Supercollider 
mit minimaler Anpassung benutzt werden. Lediglich die Rim aus Figur 4 muss, wegen ihres hölzernen Sounds, selbst 
aufgenommen werden.\\
\noindent \textbf{Figur 1}\\
Sound BD: Base Drum wird mit zunehmender dauer lauter gespielt\\
Sound HH: wird Anfangs nur sanft angespielt aber mit zunehmender Zeit etwas lauter\\
Problem: Base Drum und High Heads müssen mit zunehmender Vorführungszeit lauter werden\\ 
Lösung:\\
\begin{lstlisting}
Code: d1 $ slow 2 $ sound "[bd hh]*8" #gain "<0.7 0.9 1.1 1.3>"
\end{lstlisting}

Fig4:\\
Sound Rim: \\
Nico: selber bauen da keine hölzernen klänge vorhanden sind

Fig6:\\
irgendwas mit gain?\\



Töne: hh, bd (dumpf, wenig knackig)\\
rim

\subsection{Instrument 2: Pauken}
\subsubsection{Figuren}
{\color{red}\textbf{OFFEN}} \\
{\color{red}\textbf{TODO}}
(Hierzu Studio-Version hören)\\

\subsubsection{Klangbild}
Bestandteile der Pauke: 3 Kesselpauken\\
Art der Synthetisierung:?\\


\subsection{Instrument 3: Marimba}
\subsubsection{Figuren}
\textit{Abschnitt bearbeitet von: Raphael Drechsler}\\

\noindent \textbf{Figur 1}\\
Die Tonhöhe war per Gehör nicht genau differenzierbar. Es wurde folgende Annahme getroffen.
\begin{figure}[h]
	\centering 
	\includegraphics[width=0.5\columnwidth]{Marimba_Fig1} 
	\caption{Marimba Figur 1}
\end{figure}

\noindent Die Tonhöhe des Sounds \verb|glasstrap| wurde mithilfe des \verb|midinote|-Ausdrucks \cite{tid6} auf die jeweils gewünschte Tonhöhe geändert.
\begin{lstlisting}
p1 $ slow 2 $ midinote "[~ [51 49] 51 51][~ [51 46] 51 51][~ [51 49] 51 51][~ 51 ~ 51]" # s "glasstap"
\end{lstlisting}


\noindent \textbf{Figur 2}\\
In der Live-Version des Liedes ist im Video zu erkennen, dass auf der Marimba Kuhglocken (8 oder 10 Stück) in verschiedenen Tonhöhen liegen. Diese werden in der zweiten Figur auf die in der folgenden Abbildung gezeigten Zählzeit angespielt. Dabei wird Immer eine andere Tonhöhe gespielt. Das Muster in welcher Reihenfolge die Glocken gespielt werden wurde nicht weiter analysiert. Stattdessen soll das Anspielen der Glocken zur Vereinfachung randomisiert werden.
\begin{figure}[h]
	\centering 
	\includegraphics[width=0.25\columnwidth]{Marimba_Fig2} 
	\caption{Marimba Figur 2}
\end{figure}

\noindent Der Ausdruck \verb|irand 10| \cite{tid7} liefert einen zufälligen int-Wert von 0 bis einschließlich 9 zurück. Dieser Wert verändert in Verbindung mit dem \verb|# speed|-Ausdruck\cite{tid8} die Abspielgeschwindigkeit des \verb|can|-Samples und damit dessen Tonhöhe. So werden 9 verschiedene Kuhglocken simuliert.
\begin{lstlisting}
p1 $ stack [
slow 2 $ midinote "[~ [51 49] 51 51][~ [51 46] 51 51][~ [51 49] 51 51][~ 51 ~ 51]" # s "glasstap",
slow 2 $ midinote "[~[~~~60]~~][]" # s "can" # speed (1 + (irand 10)*0.2)
]
\end{lstlisting}

\noindent \textbf{Figur 3}\\
Es wurde folgende Figur per Gehör ermittelt. Dabei sind die Tonhöhen wie in Figur 1 angenommen.
\begin{figure}[h]
	\centering 
	\includegraphics[width=0.5\columnwidth]{Marimba_Fig3} 
	\caption{Marimba Figur 3}
\end{figure}

\noindent Umsetzung in Tidal:
\begin{lstlisting}
p1 $ midinote "[~ 49 51 49]*4" # s "glasstap"
\end{lstlisting}


\subsubsection{Klangbild}
Bestandteile der Marimba: Eine Marimba, wobei Kuhglocken zum zufälligem Anspielen enthalten sind.
Art der Synthetisierung: Da keine hölzernen Klänge im SuperCollider enthalten sind müssen diese selbst aufgenommen werden. Dies geschieht mittels 
Holzsticks die auf einen hölzernen Untergrund geschlagen werden.\\
\noindent \textbf{Figur 1}\\
Sound Marimba: Die Marimba wird am Anfang des Stücks relativ dominant, im gegensatz zu den restlichen Instrumenten gespiel\\
Lösung:\\
\begin{lstlisting}

\end{lstlisting}
\noindent \textbf{Figur 2}\\
Sound Marimba: Wird nur schwach angespielt wodurch sie in diesem Teil in den Hintergrund tritt.
Holzsticks auf Marimba in verschiednenen Tonhöhen wobei eher rhythmisch als melodisch eingesetzt, dazu Kuh-Glocken bereitstellen für Random-Funktion

\subsection{Instrument 4: Tuba}
\subsubsection{Figuren}
\textit{Abschnitt bearbeitet von: Raphael Drechsler}\\

\noindent\textbf{Figur 1}\\
Figur über einen Takt. Schlag auf Tuba-Mundstück als rhythmisches Element auf zweite Zählzeit im Takt.\\
\begin{figure}[h]
	\centering 
	\includegraphics[width=0.25\columnwidth]{Tuba_Fig1} 
	\caption{Tuba Figur 1}
\end{figure}

\begin{lstlisting}
d1 $ sound "[~ sn ~ ~]"
\end{lstlisting}

\subsubsection{Figuren}
\textbf{Figur 2}\\
Figur über 2 Takte. Instrumentalist bläst in die Tuba ohne dass die Lippen vibrieren, um ein Rauschen zu erzeugen. Pause am Ende der Figur als Atempause angenommen. 

\begin{figure}[h]
	\centering 
	\includegraphics[width=0.25\columnwidth]{Tuba_Fig2} 
	\caption{Tuba Figur 2}
\end{figure}

\begin{lstlisting}
--Idee: sound, der 2 Takte dauert alle 2 Takte 1x anpsielen
d1 $ sound "blasesoundTuba"
\end{lstlisting}

\noindent\textbf{Figur 3}\\
Wie Figur 1, hier allerdings kurzes tonloses Pusten stoßweise gespielt anstelle von Schlag auf Mundstück.\\

\noindent\textbf{Figur 4}\\
Tiefe Töne durch Tuba. Die Tonhöhe ist nahezu nicht erkennbar. Die Tonhöhe wurde jedoch durch die Analyse in Figur 6 klar. Die Töne werden langsam lauter.\\

\begin{figure}[h]
	\centering 
	\includegraphics[width=0.7\columnwidth]{Tuba_Fig4} 
	\caption{Tuba Figur 4}
\end{figure}

{\color{red}\textbf{TODO}} Gain Automation erklären.
Hall erklären.
\begin{lstlisting}
d1 $ slow 16 $ midinote "[43 38 36 38]*2" # s "superpiano" #cut 1 # room 0.85 # sz 0.8 # orbit 1 #gain "0.4 0.45 0.5 0.55 0.6 0.7 0.8 0.9"
\end{lstlisting}


\noindent\textbf{Figur 5}\\
Wie Figur 4, kräftig ausgespielt.\\

\noindent\textbf{Figur 6}\\
Wie Figur 5, maximal kraftvoll ausgespielt. Eine Oktave höher gespielt daher Tonhöhe der einzelnen Töne gut erkennbar.\\
\begin{lstlisting}
d1 $ slow 8 $ midinote "55 50 48 50" # s "superpiano" #cut 1 # room 0.85 # sz 0.8 # orbit 1 #gain 0.9
\end{lstlisting}

\subsubsection{Klangbild}
\textit{Abschnitt bearbeitet von: Nico Mehlhose}\\

\noindent Bestandteile der Tuba: normale Tuba, welche aber entartet benutzt wird\\
Art der Synthetisierung: Da die Tuba ein reales Musikinstrument ist, welches in dieser Form nicht im Supercollider enthalten ist,
werden hierfür Samples benutzt. Die Samples werden für die entartete Benutzung in Tidal so manipuliert, dass sie die Sounds
nachempfinden.\\
Für Figur 2 werden eigene Samples aufgenommen. Den Sound soll ein Handscheibenwischer mit einem hohlen Griff erzeugen.

\noindent\textbf{Figur 1}\\
 Sound: In dieser Figur wird auf das Mundstück der Tuba geschlagen. Der erzeugte Ton hört sich in etwa an wie eine Base Drum ohne Bass.\\
Lösung:\\
\begin{lstlisting}
d1 $ sound "~ bd ~ ~" # midinote 15
\end{lstlisting}

\noindent\textbf{Figur 2}\\
Sound: ähnlich eines Reifens der Luft verliert\\
Lösung:\\
\begin{lstlisting}
Nico: d1 $ sound "[sax ~ ~ hh]" # speed 0.35 # midinote 55 # gain "[1 0]" # cut 1 \\
d1 $ sound "[trump ~ ~ hh]" # speed 0.05 # midinote 55 # gain "[0.7 0]" # cut 1\\
\end{lstlisting}

\noindent\textbf{Figur 3}\\
Sound: Wie Sound aus Figur 2 aber mit mehr Druck und nicht durchgehend.\\
Lösung:\\


\subsection{Instrument 5: Posaune}
\subsubsection{Figuren}
\textit{Abschnitt bearbeitet von: Raphael Drechsler}\\

\noindent\textbf{Figur 1}\\
Figur über einen Takt. Kurzes, tonloses Pusten in die Posaune. Stoßweise gespielt als rhythmisches Element auf letzte Achtelnote im Takt.\\
\begin{figure}[h]
	\centering 
	\includegraphics[width=0.25\columnwidth]{Posaune_Fig1} 
	\caption{Posaune Figur 1}
\end{figure}

\begin{lstlisting}
d1 $ sound "[][[][~ sn]]"
\end{lstlisting}


\noindent\textbf{Figur 2}\\
Erzeugen von Rauschen analog zu Figur 2 - Tuba. Dabei Lautstärke zum Ende des Stückes hin zunehmend.
\begin{lstlisting}
--Idee: sound, der 2 Takte dauert alle 2 Takte 1x anpsielen
--Frage: Lautstaerke?
d1 $ sound "blasesoundPosaune"
\end{lstlisting}

\subsubsection{Klangbild}
Bestandteile der Posaune: normale Posaune, welche aber entartet benutzt wird\\
Art der Synthetisierung: Da die Posaune ein reales Musikinstrument ist, welches in dieser Form nicht im Supercollider enthalten ist,
werden hierfür Samples benutzt. Die Samples werden für die entartete Benutzung in Tidal so manipuliert, dass sie die Sounds
nachempfinden.\\
Für Figur 2 werden eigene Samples aufgenommen. Den Sound soll ein Handscheibenwischer mit einem hohlen Griff erzeugen.

\noindent\textbf{Figur 1}\\
 Sound: In dieser Figur wird auf das Mundstück der Posaune stoßartig angespielt.\\
Lösung:\\
\begin{lstlisting}
\\
\end{lstlisting}

\noindent\textbf{Figur 2}\\
Sound: ähnlich zu Figur 3 der Tuba allerdings mit weniger tief.\\
Lösung:\\
\begin{lstlisting}
\\
\end{lstlisting}

\subsection{Instrument 6: Violine}
\subsubsection{Figuren}
\textit{Abschnitt bearbeitet von: Raphael Drechsler}\\

\noindent\textbf{Figur 1}\\
Leicht schrill gespielte Figur mit gebrechlicher, wimmernder Wirkung. Die Figur wird gegen Ende Lauter und steigt in der Tonhöhe und erzeugt somit einen anwachsenden Spannungsbogen im Hörgefühl.
\begin{figure}[h]
	\centering 
	\includegraphics[width=0.99\columnwidth]{Violin_Fig1} 
	\caption{Violine Figur 1}
\end{figure}

\noindent Die Noten der Figur werden oktaviert gespielt. Der Einfachheit halber wurden nur dir gut hörbaren hohen töne umgesetzt.

\begin{lstlisting}
p "i6" $ slow 16 $ fastcat [
midinote "[82][][][][][][][75][74][82][][][][][][]" # s "gtr",
midinote "[82][][][][][][][75][74][82][][][][][][]" # s "gtr",
midinote "[82][][][][][][][75][74][86][][][][][87][86]" # s "gtr",
midinote "[][90][][][][][91][90][][93][][][][][][]" # s "gtr"
] # room 0.85 # sz 0.8 # orbit 1 #gain 0.9
\end{lstlisting}\

\noindent\textbf{Figur 2}\\
In Tonhöhe abfallende, schnelle Figur, die Bewegung erzeugt. Diese wird im Vibrato gespielt, was die Bewegung intensiviert.
\begin{figure}[h]
	\centering 
	\includegraphics[width=0.3\columnwidth]{Violin_Fig2} 
	\caption{Violine Figur 2}
\end{figure}

\begin{lstlisting}
p "i6" $ slow 2 $ midinote "[[86][84][82][[][84]]][82 81 79 78]" # s "gtr"
\end{lstlisting}

\noindent\textbf{Figur ?}\\
{\color{red}\textbf{TODO}} weitere Figuren im hinteren Teil des Stückes

\subsubsection{Klangbild}
Bestandteile der Violine: Violine\\
Art der Synthetisierung: Hierfür wurden Samples Online gesucht da es keine vergleichbaren SuperCollider Sounds gibt oder jemand eine Violine besitzt.\\
\noindent\textbf{Figur 1}\\
Sound: In dieser Figur steigt die Violine leise in das Stück ein und wird über die Zeit immer lauter, wodurch sie schlussendlich im zweiten Teil 
zum Hauptinstrument des Abschnittes wird. Im ersten Teil wird die Violine etwas quitschend und langsam gespielt. Im zweiten Teil wird sie hektisch und sauber gespielt.\\
Problem: Die Violine muss in unserer Vorführung diesen gleichmäßigen Anstieg der Lautstärke vollführen, ohne merkliche Sprünge zu machen.\\
Lösung:\\
\begin{lstlisting}
\\
\end{lstlisting}
\noindent\textbf{Figur 2}\\
Sound: Wenn die Violine in diese Figur übergeht ist sie das Hauptinstrument des Stückes. Die Violine wird hierbei sehr hektisch aber im gegensatz zu Figur 1 sauber gespielt.\\
Lösung:\\
\begin{lstlisting}
\\
\end{lstlisting}



\subsection{Instrument 7: Chello}
\subsubsection{Figuren}
\textit{Abschnitt bearbeitet von: Raphael Drechsler}\\

\noindent\textbf{Figur 1}\\
Untermalt die bewegungsvolle Figur 2 der Violine. Ist dabei ebenso bewegungsvoll und ebenfalls mit Vibrato gespielt um den Effekt zu intensivieren.
\begin{figure}[h]
	\centering 
	\includegraphics[width=0.3\columnwidth]{Chello_Fig1} 
	\caption{Chello Figur 1}
\end{figure}

\begin{lstlisting}
p "i7" $ slow 2 $ midinote "[[74][][][[][75]]][74 72 75 74]" # s "gtr"
\end{lstlisting}

\noindent\textbf{Figur 2}\\
{\color{red}\textbf{TODO}} \\

\noindent\textbf{Figur ?}\\
{\color{red}\textbf{TODO}} weitere Figuren im hinteren Teil des Stückes

\subsubsection{Klangbild}
Ruhige Parts, Hektik (schnell gespielte Töne)


\subsection{Instrument 8: Harfe}
\subsubsection{Figuren}
\textit{Abschnitt bearbeitet von: Raphael Drechsler}\\

\noindent\textbf{Figur 1}\\
{\color{red}\textbf{TODO}} Starke Annahme - hier vllt was mit Random aus Skala nehmen machen.
\begin{figure}[h]
	\centering 
	\includegraphics[width=0.3\columnwidth]{Harp_Fig1} 
	\caption{Harfe Figur 1}
\end{figure}

\begin{lstlisting}
d1 $ slow 2 $ stack [
	midinote "[74 72]*4" # s "gtr", 
	midinote "[62 60]*4" # s "gtr", 
	slow 2 $ fastcat [
		midinote "70 69" # s "gtr", 
		midinote "67" # s "gtr",
		midinote "70 69" # s "gtr", 
		midinote "66" # s "gtr"    
	]
]
\end{lstlisting}



\subsubsection{Klangbild}
Bestandteile der Harfe: Harfe\\
Art der Synthetisierung: Die Harfe wird mit Samples synthetisiert da diese nicht über die SuperCollidertöne synthetisierbar ist.\\
Sound: Die Harfe wird in ihren Teilen sehr deutlich gespielt. Jedoch schwankt die Lautstärke mit der sie gespielt wird von leise zu laut und wieder zurück.\\
Problem: Die schwankende Lautstärke ist das Hauptproblem bei der Aufführung.\\
Lösung:\\
\begin{lstlisting}
\\
\end{lstlisting}


\subsection{Instrument 9: Flügel}
\subsubsection{Figuren}
\textit{Abschnitt bearbeitet von: Raphael Drechsler}\\

\noindent\textbf{Figur 1}\\
Nur ein Tiefer Ton.
\begin{figure}[h]
	\centering 
	\includegraphics[width=0.3\columnwidth]{Flug_Fig1} 
	\caption{Flügel Figur 1}
\end{figure}

\begin{lstlisting}
p "i9" $ slow 2 $ stack[midinote "43 " #s "superpiano", midinote "55 " #s "superpiano"]
\end{lstlisting}

\noindent \textbf{Figur 2}\\
Zwei gleichzeitig, kurz angespielte Töne.
\begin{figure}[h]
	\centering 
	\includegraphics[width=0.3\columnwidth]{Flug_Fig2} 
	\caption{Flügel Figur 2}
\end{figure}\\
{\color{red}\textbf{TODO}} Code\\


\noindent \textbf{Figur 3}\\
{\color{red}\textbf{TODO}}Beschreibung\\
\begin{figure}[h]
	\centering 
	\includegraphics[width=0.8\columnwidth]{Flug_Fig3} 
	\caption{Flügel Figur 3}
\end{figure}
\begin{lstlisting}
p "i9" $ slow 2 $ midinote "[86 ~][86][~] [~] " # s "superpiano"  # room 0.5 # sz 0.83 # orbit 1 #gain "<0.65 0.7 0.75 0.8>"
\end{lstlisting}


\noindent \textbf{Figur 4}\\
{\color{red}\textbf{TODO}}Beschreibung\\
\begin{figure}[h]
	\centering 
	\includegraphics[width=0.9\columnwidth]{Flug_Fig4a} 
	\includegraphics[width=0.9\columnwidth]{Flug_Fig4b} 
	\includegraphics[width=0.9\columnwidth]{Flug_Fig4c} 
	\includegraphics[width=0.9\columnwidth]{Flug_Fig4d} 
	\caption{Flügel Figur 4}
\end{figure}
\begin{lstlisting}
d1 $ slow 8 $ fastcat [
midinote "~ 67 69 74 ~ 67 69 74 ~ 67 69 74 ~ 67 69 74 ~ 67 69 74 ~ 67 69 74 ~ 67 69 74  ~ ~ 67 ~ " # s "superpiano",
midinote "~ 67 68 69 ~ 67 68 69 ~ 67 68 69 ~ 67 68 69 ~ 67 68 69 ~ 67 68 69 ~ 67 68 69  ~ ~ 67 ~ " # s "superpiano",
midinote "~ 66 67 72 ~ 66 67 72 ~ 66 67 72 ~ 66 67 72 ~ 66 67 72 ~ 66 67 72 ~ 66 67 72  ~ ~ 67 ~ " # s "superpiano",
midinote "~ 66 67 70 ~ 66 67 70 ~ 66 67 70 ~ 66 67 70 ~ 66 67 70 ~ 66 67 70 ~ 66 67 70  ~ ~ 66 ~ " # s "superpiano"
]
\end{lstlisting}

\noindent \textbf{Figur 5}\\
{\color{red}\textbf{TODO}}Beschreibung\\
\begin{figure}[h]
	\centering 
	\includegraphics[width=0.9\columnwidth]{Flug_Fig5} 
	\caption{Flügel Figur 5}
\end{figure}
\begin{lstlisting}
d1 $ slow 2 $ midinote "~ 67 69 74 ~ 67 69 74 ~ 67 69 74 ~ 67 69 74 ~ 67 69 74 ~ 67 69 74 ~ 67 69 74  ~ ~ 67 ~ " # s "superpiano"
\end{lstlisting}


\subsubsection{Klangbild}
Bestandteile des Flügels: Flügel\\
Art der Synthetisierung:?\\
\noindent \textbf{Figur 1}\\
Sound: Der Sound der Figur wird zur Einleitung in den ersten Hauptabschnitt benutzt. Dabei werden die Tasten des Flügels 
schnell gedrückt um einen möglichst lauten Ton hervorzubringen.\\
Lösung:\\
\begin{lstlisting}
\\
\end{lstlisting}
\noindent \textbf{Figur 2}\\
Sound: Der Flügel wird in dieser Figur normal angespielt und führt den Zuhörer zu dem Höhepunkt des ersten Haupteiles welcher von der Violine gespielt wird.\\
Lösung:\\
\begin{lstlisting}
\\
\end{lstlisting}
\noindent \textbf{Figur 4}\\
Sound: normal angespielter Flügel\\
Lösung:\\
\begin{lstlisting}
\\
\end{lstlisting}

\subsection{Instrument 10: Moog Syntheziser}
\subsubsection{Figuren}
\textit{Abschnitt bearbeitet von: Raphael Drechsler}\\

\noindent\textbf{Figur 1}\\
Basslauf über 8 Takte.\\
{\color{red}\textbf{TODO}}Bessere Beschreibung\\
\begin{figure}[h]
	\centering 
	\includegraphics[width=1\columnwidth]{Bass_Fig1} 
	\caption{Moog Figur 1}
\end{figure}

\begin{lstlisting}
---Arbeitsstand
--2Takte
d1 $ midinote "[[55 55][54 55 ~ ~]]" # s "moog" # cut 1
--2Takte
d1 $ midinote "[[50 50][49 50 ~ ~]]" # s "moog" # cut 1
--2Takte
d1 $ midinote "[[48 48][47 48 ~ ~]]" # s "moog" # cut 1
--1 Takte
d1 $ midinote "[[58 57][56 57 ~ ~]]" # s "moog" # cut 1
--1 Takte
d1 $ midinote "[[57 ~][56 57 ~ ~]]" # s "moog" # cut 1

---Verbunden:
d1 $ slow 8 $ fastcat [midinote "[[55 55][54 55 ~ ~]]*2" # s "moog" # cut 1,
midinote "[[50 50][49 50 ~ ~]]*2" # s "moog" # cut 1,
midinote "[[48 48][47 48 ~ ~]]*2" # s "moog" # cut 1,
midinote "[[58 57][56 57 ~ ~]]  [[57 ~][56 57 ~ ~]]" # s "moog" # cut 1
]
\end{lstlisting}

\noindent\textbf{Figur 2}\\
Basslauf über einen Takt.\\
{\color{red}\textbf{TODO}}Beschreibung\\
\begin{figure}[h]
	\centering 
	\includegraphics[width=0.3\columnwidth]{Bass_Fig2} 
	\caption{Moog Figur 2}
\end{figure}

\begin{lstlisting}
---Arbeitsstand
d1 $ midinote "[[[[50 ~ ~ 50]][~ 50]][55 57 60 58]]" # s "moog" # cut 1
\end{lstlisting}



\subsubsection{Klangbild}
Bestandteile des Moogs: Keyboard, PC, Mischboard\\
Art der Synthetisierung: Der Moog wird von uns selbst im SuperCollider Programmiert, da ein Moog relativ leicht selbst zu Programmieren ist.
Des weiteren wollen wir uns damit die Option offen halten anstatt eines Moogs eine Bassline zu benutzen.\\ \\

Für die Synthetisierung des Moogs mussten einige kleine Teilschritte unternommen werden. Zuerst die Syntetisierung des Moogs im SuperCollider.\\
Dazu wurde folgender Code geschrieben:
\begin{lstlisting}
x=(
SynthDef(\moog, {
	arg freq=102, width=0.5, mul=0.5, freq2=300, q=0.2, mode=0;
	var moog ;
	
		moog=BMoog.ar(Pulse.ar(freq,width,  mul),freq2, q, mode, mul:0.2);
		
		
		Out.ar(0, moog);
		Out.ar(1, moog);
	
}).play
)
\end{lstlisting}
Als nächstes muss dieser Sound aus dem SuperCollider aufgenommen werden. Dies geht mit dem Befehl \textit{Server.default.record;}.\\
Der dritte und letzte Schritt war das zuschneiden der Audisamples. Dazu wurde Audacity benutzt. Danach kann dann der Code für die einzelnen Figuren angepasst werden.\\
\noindent\textbf{Figur 1}\\
Lösung:\\
\begin{lstlisting}
p "i10" $ slow 8 $ fastcat [sound "[[BBFMoog:5 BBFMoog:5][BBFMoog:4 BBFMoog:5 ~ ~]]*2" # cut 1 ,
sound "[[BBFMoog:3 BBFMoog:3 ][BBFMoog:2 BBFMoog:3 ~ ~]]*2" # cut 1,
sound "[[BBFMoog:1 BBFMoog:1 ][BBFMoog:0 BBFMoog:1 ~ ~]]*2" # cut 1,
sound "[[BBFMoog:7 BBFMoog:6 ][BBFMoog:5 BBFMoog:6 ~ ~]]  [[BBFMoog:6 ~][BBFMoog:5 ~ BBFMoog:6 ~ ~]]" # cut 1
] #gain 1.5
\end{lstlisting}
\noindent\textbf{Figur 2}\\
Lösung:\\
\begin{lstlisting}
p "i10" $ sound "[[[[BBFMoog:3 ~ ~ BBFMoog:3]][~ BBFMoog:3]][BBFMoog:5 BBFMoog:6 BBFMoog:8 BBFMoog:7]]" # cut 1 # gain 1.5
\end{lstlisting}

\section{Performance}
TODO:\\
Spuren per Stack verbinden?\\
Konzept für Ablauf Performance?\\
Wie mehrtaktige Figuren mit every zusammenfassen?
Link zu den Soundfiles: http://virtualplaying.com/virtual-playing-orchestra/ \\
https://rhythm-lab.com/moog-rogue-bass/

\pagebreak

%----------------------------------------------------------------------------------------
%	BIBLIOGRAPHY
%----------------------------------------------------------------------------------------

\renewcommand{\refname}{\spacedlowsmallcaps{Literatur/Quellen}} % For modifying the bibliography heading

\bibliographystyle{unsrt}

\bibliography{biblo.bib} % The file containing the bibliography

%----------------------------------------------------------------------------------------

\end{document}