\documentclass[
10pt, % Main document font size
a4paper, % Paper type, use 'letterpaper' for US Letter paper
oneside, % One page layout (no page indentation)
%twoside, % Two page layout (page indentation for binding and different headers)
headinclude,footinclude, % Extra spacing for the header and footer
BCOR5mm, % Binding correction
]{scrartcl}

\usepackage{listings}
\usepackage{color}
%\usepackage{biblatex}

\definecolor{dkgreen}{rgb}{0,0.6,0}
\definecolor{gray}{rgb}{0.5,0.5,0.5}
\definecolor{mauve}{rgb}{0.58,0,0.82}

\lstset{frame=tb,
	language={},
	aboveskip=3mm,
	belowskip=3mm,
	showstringspaces=false,
	columns=flexible,
	basicstyle={\small\ttfamily},
	numbers=none,
	numberstyle=\tiny\color{gray},
	keywordstyle=\color{blue},
	commentstyle=\color{dkgreen},
	stringstyle=\color{mauve},
	breaklines=true,
	breakatwhitespace=true,
	tabsize=3
}

\usepackage{german}


%usepackage[utf8]{inputenc}
%\usepackage{geometry}
\usepackage[german,onelanguage,linesnumbered, ruled]{algorithm2e}
\SetAlFnt{\small}
\SetAlCapFnt{\large}
\SetAlCapNameFnt{\large}
%\usepackage{algpseudocode}


\input{structure.tex} % Include the structure.tex file which specified the document structure and layout

\hyphenation{Fortran hy-phen-ation} % Specify custom hyphenation points in words with dashes where you would like hyphenation to occur, or alternatively, don't put any dashes in a word to stop hyphenation altogether

%----------------------------------------------------------------------------------------
%	TITLE AND AUTHOR(S)
%----------------------------------------------------------------------------------------

\title{\normalfont\spacedallcaps{Projektaufgabe AE}} % The article title

\subtitle{Remove Duplicates - Spotify playlist cleaner} % Uncomment to display a subtitle

\author{\spacedlowsmallcaps{Raphael Drechsler}} % The article author(s) - author affiliations need to be specified in the AUTHOR AFFILIATIONS block

\date{} % An optional date to appear under the author(s)

%----------------------------------------------------------------------------------------

\begin{document}

%----------------------------------------------------------------------------------------
%	HEADERS
%----------------------------------------------------------------------------------------

\renewcommand{\sectionmark}[1]{\markright{\spacedlowsmallcaps{#1}}} % The header for all pages (oneside) or for even pages (twoside)
%\renewcommand{\subsectionmark}[1]{\markright{\thesubsection~#1}} % Uncomment when using the twoside option - this modifies the header on odd pages
\lehead{\mbox{\llap{\small\thepage\kern1em\color{halfgray} \vline}\color{halfgray}\hspace{0.5em}\rightmark\hfil}} % The header style

\pagestyle{scrheadings} % Enable the headers specified in this block

%----------------------------------------------------------------------------------------
%	TABLE OF CONTENTS & LISTS OF FIGURES AND TABLES
%----------------------------------------------------------------------------------------

%\maketitle % Print the title/author/date block
{ \centering
{ \par}\
 \linebreak
\linebreak 
\linebreak
\linebreak
\linebreak
%\centering
\includegraphics[width=0.55\columnwidth]{htwLogo} 
\linebreak
\linebreak
\linebreak
\linebreak 
 % inline
{\fontsize{14}{16}\selectfont \center Fakultät Informatik, Mathematik und\\Naturwissenschaften\\Studiengang Informatik Master\par}\
 \linebreak
{\fontsize{18}{20}\selectfont \center \textbf{Projektarbeit zur Vorlesung Computermusik}\par}\
{\fontsize{20}{22}\selectfont \center \textbf{BrandtBrauerFrick.hs} \par}\
\linebreak
\linebreak
\linebreak
\linebreak 
\linebreak
\linebreak 
\linebreak 
{\fontsize{14}{16}\selectfont  \begin{tabular}{rl}
 	\textbf{Autoren:} & Nico Mehlhose, Raphael Drechsler\\ 
 	\textbf{Abgabedatum:} & ??.??.2019 \\ 
 \end{tabular}
\par}
\par}
\pagebreak
\setcounter{tocdepth}{2} % Set the depth of the table of contents to show sections and subsections only

%\tableofcontents % Print the table of contents
%\listoffigures % Print the list of figures
%\listoftables % Print the list of tables




%----------------------------------------------------------------------------------------

\newpage % Start the article content on the second page, remove this if you have a longer abstract that goes onto the second page

%----------------------------------------------------------------------------------------
%	INTRODUCTION
%----------------------------------------------------------------------------------------
\section{Abstract}\
\textbf{BrandtBrauerFrick.hs}

\noindent Brandt Brauer Frick ist ein Techno-Projekt aus Berlin.
Die Basis des Projekts bilden Klänge aus dem Instrumentarium der
klassischen Musik, welche anfangs gesampelt, später in einem zehnköpfigen
Ensemble auch live vorgeführt wurden.\cite{Wiki}\\ 

\noindent Ziel des Projektes:\\
Die Umsetzung des Songs ''Pretend'' von Brandt Brauer Frick entweder in
Tidal oder Euterpea. Dabei Orientierung an der Live-Aufführung
(https://www.youtube. com/watch?v=KCpLXpMB7F8).\\


\noindent Herausforderungen:
\begin{itemize}
	\item Evaluation ob Tidal oder Euterpea genutzt werden soll:
	\item Untersuchung der Frage ob klassische Klänge am ehesten in Euterpea oder
	Tidal nutzbar sind. (Durch repetitiven Charakter des Liedes würde sich Tidal zur
	Live-Vorführung eignen)
	\item Analyse der einzelnen musikalischen Bausteine und deren Implementierung.
\end{itemize}

\section{Tidal oder Euterpea}\
\textbf{Tidal oder Euterpea?}

\noindent Welche Klänge sind zu für die zehn Instrumente ungefähr/ im allgemeinen zu erwarten? In Tidal oder Euterpea besser abbildbar?
Dazu Berücksichtigung dass sich Tidal zur Live-Vorführung eignen würde.


\section{Analyse des Stücks Pretend}\
Im folgenden Abschnitt sollen die zehn Instrumentalisten untersucht werden.

\subsection{Instrument 1: Schlagzeug}
\textbf{Instrument}\\
...\\
\textbf{Gespielte Elemente}\\
...\\
\subsection{Instrument 2: Pauken}
\textbf{Instrument}\\
...\\
\textbf{Gespielte Elemente}\\
...\\
\subsection{Instrument 3: Marimba}
\textbf{Instrument}\\
...\\
\textbf{Gespielte Elemente}\\
...\\
\subsection{Instrument 4: Tuba}
...
\subsection{Instrument 5: Posaune}
\subsection{Instrument 6: Violine}
\subsection{Instrument 7: Chello}
\subsection{Instrument 8: Harfe}
\subsection{Instrument 9: Flügel}


\subsection{Instrument 10: Moog Syntheziser}
\textbf{Figur 1}\\
Basslauf über 8 Takte\\

\noindent\textbf{Figur 2}\\
Basslauf über einen Takt


\section{Vielleicht ein Kapitel zur Umsetzung?}


\pagebreak

%----------------------------------------------------------------------------------------
%	BIBLIOGRAPHY
%----------------------------------------------------------------------------------------

\renewcommand{\refname}{\spacedlowsmallcaps{Literatur/Quellen}} % For modifying the bibliography heading

\bibliographystyle{unsrt}

\bibliography{biblo.bib} % The file containing the bibliography

%----------------------------------------------------------------------------------------

\end{document}